\documentclass{my_paper}
\usepackage{ctex}
\usepackage[textwidth=444bp,vmargin=2.5cm]{geometry}%设置页边距
\usepackage{array} %主要是增加列样式选项
\usepackage[dvipsnames]{xcolor}%颜色宏包
\usepackage{graphicx}%图片宏包
\usepackage{amsmath}%公式宏包
\usepackage{listings}
\lstset{language=Matlab}
\usepackage[UTF8]{ctex} % 显示中文
\usepackage{xcolor} 
\usepackage{listings}
\lstset{%
	alsolanguage=Java,
	%alsolanguage=[ANSI]C,      %可以添加很多个alsolanguage,如alsolanguage=matlab,alsolanguage=VHDL等
	alsolanguage= matlab,
	alsolanguage= XML,
	tabsize=4, %
	frame=shadowbox, %把代码用带有阴影的框圈起来
	commentstyle=\color{red!50!green!50!blue!50},%浅灰色的注释
	rulesepcolor=\color{red!20!green!20!blue!20},%代码块边框为淡青色
	keywordstyle=\color{blue!90}\bfseries, %代码关键字的颜色为蓝色,粗体
	showstringspaces=false,%不显示代码字符串中间的空格标记
	stringstyle=\ttfamily, % 代码字符串的特殊格式
	keepspaces=true, %
	breakindent=22pt, %
	numbers=left,%左侧显示行号 往左靠,还可以为right,或none,即不加行号
	stepnumber=1,%若设置为2,则显示行号为1,3,5,即stepnumber为公差,默认stepnumber=1
	%numberstyle=\tiny, %行号字体用小号
	numberstyle={\color[RGB]{0,192,192}\tiny} ,%设置行号的大小,大小有tiny,scriptsize,footnotesize,small,normalsize,large等
	numbersep=8pt,  %设置行号与代码的距离,默认是5pt
	basicstyle=\footnotesize, % 这句设置代码的大小
	showspaces=false, %
	flexiblecolumns=true, %
	breaklines=true, %对过长的代码自动换行
	breakautoindent=true,%
	breakindent=4em, %
	escapebegin=\begin{CJK*}{GBK}{hei},escapeend=\end{CJK*},
	aboveskip=1em, %代码块边框
	tabsize=2,
	showstringspaces=false, %不显示字符串中的空格
	backgroundcolor=\color[RGB]{245,245,244},   %代码背景色
	%backgroundcolor=\color[rgb]{0.91,0.91,0.91}    %添加背景色
	escapeinside=``,  %在``里显示中文
	%% added by http://bbs.ctex.org/viewthread.php?tid=53451
	fontadjust,
	captionpos=t,
	framextopmargin=2pt,framexbottommargin=2pt,abovecaptionskip=-3pt,belowcaptionskip=3pt,
	xleftmargin=4em,xrightmargin=4em, % 设定listing左右的空白
	texcl=true,
	% 设定中文冲突,断行,列模式,数学环境输入,listing数字的样式
	extendedchars=false,columns=flexible,mathescape=true
	% numbersep=-1em
}
\usepackage[T1]{fontenc}    
\usepackage{newtxtext, newtxmath}  %两种使用Times New Roman 字体的方法

\begin{document}
	\begin{table}
		\centering
		\includegraphics{format2022.pdf}
	\end{table}
	\newgeometry{left = 2.5 cm, top=2.5cm,right =2.5cm}
	%----------- 中文摘要 ----------
	\newpage
	
	\begin{center}
		\lunwenbiaoti
		
		\vspace{2ex}
		\zhaiyao
	\end{center}
	
	本文主要研究论文评审过程中公平性,合理性与可行性的问题。保证尽可能均匀的评委分配,降低评委打分偏好的影响都是保证评审公平的重要方面,解决上述相关为题是非常具有实际意义的。由于实际中问题规模往往不在可以人工处理的范围内,所以本文主要采用matlab,python编程,使用excel,python进行数据处理
	
	问题一中,采用了如下基本方法:
	
	(1)直接利用组合数定义得到,$m \leq C_n^3$
	
	(2) 该问题是实际上是施坦纳三元系问题(steiner triple system),直接利用相关结论即可
	
	(3)将问题转化为01优化问题,使用lingo求解出最优解
	
	问题二中,通过对评委打分情况进行分析,构造一种比较合理的分数调整方案,调整分数后可以得到更为公平的结果
	
	问题三中,依托于问题二的基础之上,对各个步骤进行评估,对各个步骤进行修改以适应动态变化的权系数,再根据权系数修正论文的最终评分
	
	问题四中,根据问题一二三中研究结果,引入圆桌模型以及裁判复核来使评审过程更为合理高效
	
	
	
	
	\begin{guanjianci}
		三元系 \quad 01优化 \quad 加权分数修正 \quad 动态权系数 \quad 圆桌模型 \quad 裁判复核 
	\end{guanjianci}
	
	%----------- 正文 ----------
	%----------- 一、问题重述 ----------
	\newpage
	\section{一、问题重述}
	随着信息时代的到来,科学技术日新月异。在信息不断更新、不断变革的大潮中,考试和竞赛非但没有被取代,反而成为衡量教学水平和考察学生素质的一个重要标准。但这同时也带来了许多问题,尤其突出的是考试或竞赛中评阅是否公平、经济、合理的问题。因此,必须建立出一个较为客观的判断及改进评阅公正性的数学模型。从实际情况出发,根据不同的评审制度,解决如下问题:
	
	(1)若参加竞赛共有m组,n个评委,每组论文3位评委评审,给出论文与评委的分配方法,找出任意不同的两个组的评委尽量不同时m、n值的满足条件。
	
	(2)在评委打分偏紧偏松的情况下,若10个评委评阅所有论文,请你给出数学模型与算法,估计每位评委打分的偏差,并对评委打分进行调整,计算获奖结果。
	
	(3)在评委只评阅了部分论文的情况下,给出数学模型与算法,就评审的每个步骤将分数进行调整,并计算出获奖结果。
	
	(4)对现行评审步骤的公正性,评委的工作量安排的合理性给出数学模型对现行的评审步骤进行评价并改进
	
	%----------- 二、问题分析 ----------
	\section{二、问题分析}
	1)以第一题的第1,2小问得出的结果作为第3小问求解最优分配方案时的首要考虑因素,尽量减少任意两组有2个评委相同的情况,从而提高评委阅卷的广泛性,实现评审的公正性以及评委工作安排的合理性。
	
	(2)在评卷过程中,不可避免地会出现评委评分偏松或偏紧的情况。于是,给出一种评分一致性或公正性的检验方法,是非常必要的。运用统计学的原理,引入几个统计量,根据各评委的评分特点在这些统计量上表现出来的不同特征,对不同特点的评委进行分类,从而实现评分一致性或公正性的检验。
	
	(3)第三题中评分已经分成了4个阶段,这样,就不能原样的沿用问题二的模型,但是在假设评委态度不变的情况下,我们可以考虑,通过步骤1来确定各个评委之间的差异,以及调整方案。这样,就可以在第二题的模型基础上稍加改动,运用于第三题中。
	%----------- 三、模型假设 ----------
	\section{三、模型假设}
	1.每个评委都必须参与其中。
	
	2.每位评委在评审的整个过程中,评阅论文的风格始终保持一致。
	
	3.评委阅读论文速度相同。
	
	4.评委之间不会相互交流给分情况
	
	5.论文编号是随机分配的。
	
	
	%----------- 四、符号说明 ----------
	\newpage
	\section{四、符号说明}
	%使用三线表格最好~
	\begin{table}[htbp]%htbp表示的意思是latex会尽量满足排在前面的浮动格式,就是h-t-b-p这个顺序,让排版的效果尽量好。
		\centering
		\begin{tabular}{p{2.0cm}<{\centering}p{9.0cm}<{\centering}p{2.0cm}<{\centering}}
			%指定单元格宽度, 并且水平居中。
			\hline
			符号 & 说明 &  \\ %换行 
			\hline
			n & 评委数量 & \\
			m & 分组数量 & \\
			$[x]$ & 表示x的整数部分 & \\
			A & 评分关系矩阵 & \\
			$a_{ij}$ & 评委i是否给第j组评分 & \\
			$s_{ij}$ & 评委i给第j篇论文评分的分值 & \\
			$\overline{s_j}$ & 第j篇论文得分的均值 & \\
			$w_i$ & 评委i的权系数 & \\
			$S_j$ & 第j篇论文经过调整后的分数 & \\
			R & 重复度& \\
			R' & 一个评委参与评分的分组数量的最大值 & \\
			$S_a^i$ & 离差绝对值之和 & \\
			$m_i$ & 均值差绝对值的均值 & \\
			$n_i$ & 均值差绝对值的标准差 & \\
			$r_i$ & 均值变异系数 & \\
			C & 完成度 & \\
			Q & 匹配度 & \\
			
			\hline
		\end{tabular}
	\end{table}
	
	%----------- 五、模型的建立与求解 ----------
	\section{五、模型的建立与求解}
	
	(注意:这个部分里面的标题可根据你的论文内容进行调整,我这里给的是一个通用的模版)
	
	\subsection{问题一}
	\subsubsection{第一问建模与求解}
	任意两个组的评委不能完全相同,求解组数m和评委数n的关系。从评委角度出发。
	
	(1)根据组合数的定义,得出n个评委,每3个一组,总共能构成$C_n^3$个不完全相同的组,根据抽屉原理可知
	$$m \leq C_n^3$$
	
	(2)每个评委都要参与评分,那么极端情况为所有组的评委都完全不一样,可知
	$$n \leq 3m $$
	
	综上所述,m,n的关系为
	$$\frac{n}{3} \leq m \leq C_n^3$$
	\subsubsection{第二问的建模与求解}
	要保证任意两个组出现没有两位评委相同的情况,求m,n的关系
	(1)建模
	
	给n个评委编号为1,2,$\cdots$ ,n,并且用集合表示$J=\{ 1,2,\cdots,n \}$,问题转化为求J的所有三元素子集并且这些子集中任意两个只有一个重复元素
	
	(2)模型求解
	
	定义符号[x]表示取x的整数部分,那么每一个元素都至多在$[\frac{n-1}{2}]$个不同的子集中,那么如果计入重复,那么有$n [ \frac{n-1}{2} ]$个,而再上述计算中,每个元素会重复3次,那么可以得到一个比较紧的m的上界,即
	$$[\frac{n [ \frac{n-1}{2} ]}{3}]$$
	下面说明这是一个比较紧的上界:
	
	case1:当$n =1,3 \mod 6$时,
	
	原上界可以化简为$\frac{n(n-1)}{6}$,T.Skolem已经证明该上界是可以取到的 \cite{ref1}
	
	case2:当$n=0,2 \mod 6$时,同case1可以得出该上界是可以取到的
	
	case3:当$n=4,5 \mod 6$时,
	当n=6k+5时,记上界为S,即$S=6k^2+4k+3$
	
	根据定理1.1.3\cite{ref2},含有6n+5个元素的PBD可以为一个长为5的块和L个长为3的块,即有最大值M满足
	$$M \geq [\frac{(6k+5)(3k+2)-S}{3}] =6k^2+9k+\frac{10-S}{3}$$
	
	注意到L为整数,所以10-S是3的倍数,即S=1或4
	
	从而得出$M \geq 6k^2+9k+3$或者$M \geq 6k^2+9k+3$,即$M\geq S-1$
	
	当n=6k+4时,对比n=6k+5的情况
	
	1.M=S则去掉一个点,减少3k+2种可能,$M=6k^2+6k+1$
	
	2.M=S-1,则必有一个点的配对数为$\frac{n}{2}-1$个,那么将这个点删去将会减少3k+1个可能,此时,$\mbox{            }$ M=$6k^2+6k+1$
	
	又,对于n=6k+4,有上界$S'=6k^2+6k+1$
	
	综上所述,m和n的关系为
	$$m \leq [\frac{n [ \frac{n-1}{2} ]}{3}]$$
	只有当$n=5 \mod 6$时,该上界不是紧的,最大值最少只会比上界小1
	
	\subsubsection{第三问的建模与求解}
	(1)建模
	有m个组,n个评委,将评委给一个组评分这个关系表示成大小为$m \times n$的矩阵,其中矩阵第i行第j列的元素为1表示j号评委给第i个组评分,反之则为0
	$$
	A=\begin{pmatrix}
		a_{11} & a_{12} & \cdots & a_{1n} \\
		a_{21} & a_{22} & \cdots & a_{2n} \\
		\vdots & \vdots & \ddots & \vdots \\
		a_{m1} & a_{m2} & \cdots & a_{mn} \\
	\end{pmatrix}
	$$
	由于每个组都有且仅有3个评委评审,所以有
	$$\sum_{j=1}^{n}a_{ij}=3$$
	引入一个重复度R表示任意两个组的评委重复个数之和,对于任意m,n,可知R=3m-n,给定m,n确定值时,重复度R是确定的
	
	此外引入R'表示一个评委参与评分的分组数量的最大值
	
	结合上述两个指标,原问题转化为最优化问题,如下:
	\begin{align*}
		\quad \quad \qquad \qquad &min \quad R'=\max_{j} \sum_{i=1}^{m} a_{ij},j \in \{1,2,\cdots n\} &\\
		&s.t.    \sum_{j=1}^{n}a_{ij}=3,\forall i \in \{1,2,\cdots,n\} & 
	\end{align*}
	(2)模型求解
	带入具体数据m=20,n=10,使用lingo求解上述模型优化问题
	
	一个最优的分配方式如下:
	
	\begin{table}[htbp]
		\centering
		\includegraphics[scale=0.7]{1.3.pdf}
		\caption{分配方案}
	\end{table}
	
	\newpage
	\subsection{问题二模型的建立与求解}
	\subsubsection{数据预处理}
	观察评委打分情况
	\begin{figure}[htbp]
		\centering
		\includegraphics[scale=0.8]{pw.png}
		\caption{评委打分情况}
	\end{figure}
	根据上述图标,可以按评分风格大致分为如下几类:
	1)客观公平型:这种评委的评分线围绕平均分线做小幅波动.他们试图按照客观标准掌握评分标准,保持客观公正态度,谨慎小心,即使在平均分线上下波动,但起伏不大。从经验角度看,这类评委的评分水平比较高,主观倾向与客观实际相接近,是比较理想的评委.
	
	2)一致性偏高型:评委的评分线始终高于全体评委打分确定的平均分线,与平均分线呈近似平行的关系.这种评委是一种带有主观色彩的“公正”评委,有一个稳定的主观倾向在理解和掌握评分标准,坚持按偏松的倾向打分。
	
	3)一致性偏低型:评委的评分线始终低于全体评委打分确定的平均分线,与平均分线呈近似平行的关系.与一致性偏高型评委相似,坚持从严的倾向打分,
	
	4)大幅度波动型:这种类型评委的评分结果的波动性很严重。但均值与平均分线相差不大.
	
	5)作弊型:这种评委的评分线与平均分线的关系有明显的不规则形态.这里最常见的情况是,该评委给出的大多数对象的分数与平均分线呈一有规律的吻合,但是在少数个别分数上出现明显的跳跃,远离平均分线.似乎在有意压低其他大多数测评对象的分数,而故意抬高自己看好或是有特殊关系的个别测评对象的分数.
	\subsubsection{模型建立\cite{ref3}}
	上一部分中,根据数据变化的趋势大致分为了四类,下面就需要更为精确的堆评委进行分类
	
	
	先给出如下新变量的定义:
	
	1)离差绝对值之和$S_a^i$
	
	以每一份答卷为对象,计算所获得的10位裁判打分的平均分。考虑某一位裁判打分与上述平均值之差的绝对值,在200份答卷中求和可得离差绝对值之和。          
	$$ S_a^i=\sum_j |s_{ij}-\overline{s_j}|\quad (i=1,2,\codts,10;j=1,2,\codts,200)$$
	上式表示第个评委对第份答卷的评分,表示第份答卷的平均分.
	
	2)均值差绝对值的均值$m_i$
	
	均值差绝对值的均值描述的是该裁判打分与作品所得平均分之间的平均差距。越大就意味着这位裁判的打分在不同作品之间分差较大。
	$$m_i=\frac{S_a^i}{200}$$
	$\quad$3)均值差绝对值的标准差$n_i$
	
	均值差绝对值的标准差描述的是该裁判打分与作品所得平均分之差的离散程度,包含着裁判打分围绕平均值振荡程度大小的信息。
	
	4)变异系数
	
	变异系数是某一位裁判对于所有答卷评分结果标准差与裁判分均值之。用来描述数据的离散程度。当变异系数越大时,表明数据的离散程度越显著,也即裁判的判分分差较大,在不同作品中打分有明显差异。在裁判的分类中起到一个公平指数的作用。我们认为,公正的裁判在面对作品品质呈现正态分布的情况下所做出的判分应该是相对紧凑的,不会出现大量评分明显高于或低于作品平均分的情况。所以认为,该裁判的变异系数在某个范围内时,可判定该裁判是公平的,否则为不公平的。
	
	5)均值差变异系数$r_i$
	
	均值差变异系数是每个评委的均值差绝对值的标准差与均值差绝对值的均值的比值,
	则对第i个评委而言,
	
	有: $ r_i=\frac{n_i}{m_i} ,\quad i=1,2,\cdots,10   $
	
	与被用作公平指数的变异系数不同,均值差变异系数反映出不同的裁判打分的横向对比,保留着裁判与平均打分差异的信息。均值差变异系数越大,则意味着裁判打分的波动越强。所以当均值差变异系数在某个范围内时,可以认为该裁判的打分波动虽大但在允许范围之内,但当均值差变异系数超过该范围时,可以理解为该裁判经常会做出明显区别于其他正常裁判的判分,故认定为作弊。
	
	问题转换为按照上述标准堆各个评委的打分情况进行分析,然后进行分类
	
	计算全部评委的均值差绝对值的均值,将超过7.49的视为不公平型,将低于7,4的视为公平型。其次,对于不公平型计算均值差变异系数,以-0.5为界限将其分为大幅波动型和作弊型,对于公平型再根据均值差均值,以4和-4为标准来区分一致偏高型和一致偏低型
	
	然后在根据分类结果堆数据进行调整
	
	再引入权系数w,权系数和所属分类有关
	
	1)客观公平型,w=1
	
	2)一致偏高(低)型,及波动型:
	$$\mbox{对于评委i而言,}w_i=\frac{\displaystyle \sum_{j=1}^{n} \overline{s_{j}}}{\displaystyle \sum_{j=1}^{n}s_{ij}}$$
	3)作弊型,权系数w=0,相当于直接将评分作废
	
	那么最终得出的分数调整公式如下:
	$$S_j= \displaystyle \frac{1}{m_{eff}}\sum_{i=1}^{n}w_i s_{ij}$$
	其中$m_{eff}$表示权系数不为0的评委个数,也就是有效评分个数
	\subsubsection{模型求解}
	首先利用excel得到各个评委给分情况,然后根据设定的指标给评委分类,并计算权系数,然后调整各个文章的得分,最终得到调整后成绩和排名的情况

	\begin{figure}[htbp]
		\centering
		\includegraphics[scale=0.8]{pw1.png}
		\caption{评委1一致偏高}
	\end{figure}

	\begin{figure}[htbp]
		\centering
		\includegraphics[scale=0.8]{pw2.png}
		\caption{评委2一致偏高}
	\end{figure}
	
	\begin{figure}[htbp]
		\centering
		\includegraphics[scale=0.8]{pw3.png}
		\caption{评委3一致偏高}
	\end{figure}
	
	\begin{figure}[htbp]
		\centering
		\includegraphics[scale=0.8]{pw4.png}
		\caption{评委4公正}
	\end{figure}
	
	\begin{figure}[htbp]
		\centering
		\includegraphics[scale=0.8]{pw5.png}
		\caption{评委5一致偏低}
	\end{figure}
	\begin{figure}[htbp]
		\centering
		\includegraphics[scale=0.8]{pw6.png}
		\caption{评委6一致偏低且波动大}
	\end{figure}
	\begin{figure}[htbp]
		\centering
		\includegraphics[scale=0.8]{pw7.png}
		\caption{评委7公正}
	\end{figure}
	\begin{figure}[htbp]
		\centering
		\includegraphics[scale=0.8]{pw8.png}
		\caption{评委8一致偏高}
	\end{figure}
	\begin{figure}[htbp]
		\centering
		\includegraphics[scale=0.8]{pw9.png}
		\caption{评委9一致偏低}
	\end{figure}
	\begin{figure}[htbp]
		\centering
		\includegraphics[scale=0.8]{pw10.png}
		\caption{评委10一致偏高且波动大}
	\end{figure}	
	\newpage
	\leftline{1)公正型与非公正型}
	\begin{table}[htbp]
		\centering
		\includegraphics[scale=1.0]{faircrop.pdf}
	\end{table}\\
	\leftline{2)公正型中客观公正型、一致偏高型、一致偏低型的细致分类}
	\begin{table}[htbp]
		\centering
		\includegraphics[scale=1.0]{fairincrop.pdf}
	\end{table}\\
	\leftline{3)波动较大与作弊型分类}
	\begin{table}[htbp]
		\centering
		\includegraphics[scale=0.9]{zuobicrop.pdf}
	\end{table}\\
	\newpage
	\leftline{4)评委权系数分配情况}
	\begin{table}[htbp]
		\centering
		\includegraphics[scale=0.7]{wcrop.pdf}
	\end{table}\\
	\leftline{ 5)调整分数后成绩情况}
	\begin{table}[htbp]
		\centering
		\includegraphics[scale=0.55]{lastcrop.pdf}
		\caption{最终评奖情况}
	\end{table}\\
 	\newpage
	
	\subsection{问题三模型的建立与求解}
	
	在评委不可能评阅所有的论文的实际情况下,除了评委阅卷水平参差的偏差,由于数据量的减少,评委选择而产生的随机性,也增添了打分的偏差。因此,我们需要在问题二的数学模型基础上进行适当修正,以获得更好的精度。
	
	我们的修正方式大致如下:
	
	(1)步骤1的修正:
	
	由于在步骤1中关于评委给分情况的信息比较少,不能立即给评委加权系数,暂且把权系数均设为默认值1,所以这样设置是合理的。
	
	
	(2)步骤2的修正:
	在第二个步骤中,我们认为收集到的数据已经有了一定规模,所以在步骤二中利用同问题二中的方法给每个评委赋予不同的权重。对于每组入围论文让另两名评委评阅的分数,可根据最终分数调整公式:进行修改,由5名评审得分的均值得出120篇论文的排名,并且淘汰排名靠后的20篇论文.
	
	(3)步骤3的修正:
	依据对题意的理解,入围的排名1-15及41-60的论文都分别被5名评委评阅过了(虽然每篇论文对应的五名评委不尽相同),则参与步骤③的另两位评审从其他5名中随机抽取。并且因每篇入围论文未评的5个评审各不相同,故随机抽取要进行50次。
	
	针对每篇入围论文,随机产生参与步骤③的另两位评审后再进行评分。对于这两个分数,仍依据步骤①获得的第i位评委的权系数,仍然利用如下公式进行修正。
	$$S_j= \displaystyle \frac{1}{m_{eff}}\sum_{i=1}^{n}w_i s_{ij}$$
	原排名41-60间的论文由7名评审的均分排序,新排名41-50的获二等奖,51-60的获三等奖;原排名1-30间的论文由7名评审的均分排序,新排名26-30的获二等奖,1-30的获一等奖。
	
	(4)步骤4的修正:
	排名16-25的10篇论文(已由不同的7位评审评过),再由对应剩下的三名评审评阅。仍依据步骤①获得的第i位评委的权系数,对这三个评委的打分,分使用数调整公式:修正。最后,这原排名16-25的10篇文章,根据10个评委的修正后分数的均值重新排序,在新排名下16-20名的获一等奖,21-25的获二等奖。
	
	分数调整、平均再排序后的最终结果如表三所示:
	\begin{table}
		\centering
		\includegraphics[scale=0.9]{lla.pdf}
		\caption{最终评奖情况}
	\end{table}
	
	\newpage
	\subsection{问题四模型的建立与求解}
	先前的流程中,评审的公正性和评委的工作量均存在不合理性,所以本问题主要是对原评审流程的评价与改进
	
	 \subsubsection{新的衡量标准}
	 1)完成度C,评分系统总完成度C*
	 
	 在评卷过程中,评委总数为n,若一个对象由$X\in \{0 \leq X \leq n,X \in Z\}$个评委评阅,则定义其完成度为
	 $$C=\frac{X}{n}$$
	 依据上述定义,如果一队的论文被全部评委评阅,其置信度为1;如果只有部分评委评阅,则其置信度位于(0,1)。依据模型二的思想,从统计学意义上,一份试卷评阅次数越多,意味着评阅完成程度越高,其评分的系统误差越小,成绩更加接近于客观水平。一般来说,完成度紧密依赖于评委的类型,其对成绩的影响量和自身改变量并不一定是线性关系,但在本题中我们只关心成绩影响关于完成度的函数的单调性,因此不妨假设变化关系是线性的。
	 
	 类似的,我们可以得到一个有N个杨门的评分1系统的总完成度C*:
	 $$C*=\displaystyle \frac{\displaystyle \sum_{i=1}^{n} C_i}{n}$$
	 其中$C_i$表示第i个样本的完成度
	 
	 2)阶段匹配度Q
	 
	 在类似于本题的有n份样本,m位评委的多过程评分系统中(每轮每组评委数相同),仅凭借总完成度无法评估某个特定过程的评分情况,因此引入阶段匹配度的概念。在第k个阶段结束后,设每份作品已最多有$m_k $个评委打分,$n_k$ 份样本确定结果(淘汰或获x等奖),定义阶段匹配度如下:
	 $$Q_k=\displaystyle\frac{\frac{m_k}{m}}{\frac{n_k}{n}}$$
	 当匹配度在1左右时,表明评分过程中评委数量分配情况与样本数量变化情况较符合;匹配度$Q_k$<1,表明评委数量分配过少,样本评分的系统误差将会增大;匹配度$Q_k>1$,表明评委工作量增大。一般而言,匹配度序列应满足如下要求:
	 
	 1.评委分配应保持大致一致,即$Q_k$的极差/标准差应控制在一定范围。
	 
	 2.评分后期应更加严谨,即$Q_k$最好为单调不减序列。
	 
	 \subsubsection{原评审流程的分析和修改}
	 (一)本问中假设和先前模型假设略有不同,故重新罗列基本假设:
	 
	 1.每个评委的评卷速度相近;
	 
	 2.试卷编号随机分配;
	 
	 3.每位评委的阅卷质量恒定;(*)
	 
	 4.每个评委独立地评出每份答卷的分数,对于评阅同一份答卷的评委不会相互交流各自所评的分数。
	 
	 5.各个评委都是公正的,即权系数w均为为1。(*)
	 
	 (二)对于现有评审标准的简单评述
	 
	 那我们可以主要考虑四个方面:总完成度、阶段匹配度、评委的阅卷总量(劳动量)和单个评委的阅卷量。
	 
	 本实例中的评审步骤已经大大减少了劳动量,但因此导致了总完成度和阶段匹配度的一些问题,以下为分析:
	 
	 \begin{align*}
	 	&\mbox{劳动量}M=200 \times 3+120\times2+10\times3=970 \\
	 	&\mbox{总完成度}C*=\displaystyle \frac{80\times 0.3+70\times 0.5+40\times 0.7+10 \times 1}{200}=43.5% 
	 	~\\
	 	&Q_1=\frac{\frac{3}{10}}{\frac{80}{200}}=0.75\\
	 	&Q_2=\frac{\frac{5}{10}}{\frac{150}{200}}=0.67\\
	 	&Q_3=\frac{\frac{7}{10}}{\frac{190}{200}}=0.74\\
	 	&Q_4=\frac{\frac{10}{10}}{\frac{200}{200}}=1
	 \end{align*}
 
 	可看出,原评审流程总完成度低于50$\%$,并不能很好控制打分造成的系统误差。进一步分析表明,阶段匹配度序列振荡较大,无单调性;第一、二、三轮匹配度都明显低于1,即大多数样本在评分不充分的情况下就被淘汰或决定奖项,从而使整个比赛评判的不确定性加强,降低了比赛的客观公平性,应当给予改进。
	 	
	 (三)修正与改进
	 
	 修改一:应适当提升评委工作量以提高总完成度,并控制阶段匹配度。具体方法为:
	 
	 步骤一淘汰人数减少至60人;
	 
	 步骤二每篇文章评委人数增至4人;
	 
	 步骤三每篇文章评委人数增至3人;
	 
	 步骤四重新评分,需要每一位评委评分。
	 
	 
	 另一方面,注意到组与组之间的论文总体水平存在着显著差异。例如,组11论文水平普遍偏低,而组7普遍偏高。这样,若采取原评审方法中步骤一、二的组内评分方式和步骤一的组内淘汰方式,有可能会将水准较高的论文直接在步骤一或二时就淘汰,从而降低了公正性。(推导过程见附录)
	 
	 修改二:每个组的10篇论文根据3位评委的给分进行平均,总的200篇论文排名后,淘汰后60篇。
	 
	 未淘汰的140篇论文,再由没有评审过的4位评委进行评审(评审时采取圆桌模式),给出得分。
	 
	 注:圆桌模式即让所有评委围坐在一张桌子周围,将评委按顺时针编号。找一名工作人员(非评委)从第一位评委开始分发论文,如果该论文已经被该评委评过,则以顺时针方向传递,直到找到未评审过该论文的评委为止。一轮分配完成后,即每位评委均拿到论文评审时,同时开始评阅,由于假设2中评委评审速度相近,每轮各个评委的评审工作可视为同时开始同时结束,这就保证了最后工作量大体均衡。
	 
	 对140篇论文根据7个评委的平均给分进行总体排序,淘汰排名靠后的40篇文章,剩下100篇论文为获奖论文。其中71~100名评为三等奖,31~40评为二等奖。1~30名和41~60名进入下一轮。
	 
	 (*):实际评分中,(*)假设很难满足,由此带来两个问题:
	 
	 1)大规模评分时评委出现状态波动,导致个别评分偏离客观水平,由于样本量较大,问题二中的模型很难检测出异常并修正。
	 
	 2)对于大幅波动型的评委,模型二中的权系数对评分的修正效果有限,从而影响评分的公正性。
	 
	 \subsubsection{评审流程最终修改结果}
	 准备:从10名评委中选出两位资深评委(公正型)作为裁判组成员,两名裁判可互相交流,最终评出分数;将200份文章分成20组,每组10篇。设置“异常卷”条件,如极差大于10或标准差大于6。
	 
	 步骤(1)
	 
	 打分:每组的10篇论从余下8位评委中选取3位评委,评审每组的10篇论文,分别用百分制给出评分。若评分满足异常卷,则调至裁判组评判,用裁判组分数代替原平均分。评委的分配方式使用问题一中第三小问最优分配规划得到的分配方案。
	 
	 排序:每个组的10篇论文根据3位评委的给分进行平均,总的200篇论文排名后,淘汰后60篇。
	 
	 步骤(2)
	 
	 打分:未淘汰的140篇论文,再由没有评审过的4位评委进行评审(评审时采取圆桌模式),给出得分,异常卷同步骤(1)。
	 
	 排序:对140篇论文根据5个评委的平均给分进行总体排序,淘汰排名靠后的40篇文章,剩下的100篇论文为获奖论文。
	 
	 100篇获奖论文中,排名31到40的论文,获二等奖;排名71~100的论文获三等奖;
	 
	 步骤(3)
	 
	 打分:排名1~30与排名41~70的60篇论文,由1位未评审过的评委和裁判组两位评委分别打分(此步骤裁判不能交流)。
	 
	 排序:原排名41~70的20篇论文,计算得到均分后排名,新排名41~50的获二等奖,新排名51~70的获三等奖;
	 
	 原排名1~30的30篇论文,计算得到均分后排名,新排名1~15名获一等奖。
	 
	 步骤(4)
	 
	 打分:步骤(3)中平均分排名在16到30的论文,直接交给十位评委重新打分得到新的排名。
	 
	 排序:最终排名16~20的获一等奖,最终排名21~30的获二等奖。
	
	\section{六、模型的评价、改进与推广}
	
	\subsection{模型的优点}
	A.用于解决问题一前两问的模型均有可靠的数学依据,非常严谨,求解模型得到的最优解几乎无法再进行优化
	
	B.考虑的模型假设数量有限,建立的模型都具有一定的实际意义
	
	C.运用统计学的原理,根据不同的评分特点把评委分成不同类型,再用层次分析法来检验评委的公平性,使得问题的描述比较清晰,并且根据不同类型赋予不同权系数,更有说服力
	
	D.问题三的模型仅依赖于部分数据,最终得出的排名和依据全部数据得到的结果比较相似,说明该模型适用性较好,在带有随机性的同时保证了不错的稳定性
	
	E.问题四中引入了圆桌模型和裁判复合模型,可以保证工作的均衡性,同时也能使评审更为高效
	\subsection{模型的缺点}
	A.由于编程能力有限,采用的算法都很朴素,导致时间复杂度较高,数据量庞大时效率低下
	
	B.通过模型无法正真解决评委评分公平性的问题,模型即使有一定的实际意义,但具有更多局限性,要想从根本上解决评审公平问题,需要提高评委群体乃至整个社会的诚信观念和道德素养。
	
	C.没有考虑评委阅卷过程中疲劳程度的变化,从实际角度出发随着评卷进行,评委会更加疲劳,往往更容易草率地打出一个习惯性给出的分数,模型并没有考虑这一点
	\subsection{模型的改进}
	
	可以尝试采用更为高效的算法
	
	可以适当修改模型,减少冗余部分
	
	可以引入一些函数来模拟各种特殊因素(如评委疲劳程度)对于给分的影响,使得模型更加全面
	
	
	\subsection{模型的推广}
	本文中的模型不知可以运用于竞赛评分,多轮决策也可以使用,本文中模型的要点在于适应性,可以根据先前的评分结果胴体调整评委的权系数,使得评审更加公平。在多轮决策中,着就提心为一个人在每一轮决策中的话语权(权系数)会根据之前评分的结果不断变化,属实比较合理
	
	
	%----------- 参考文献 ----------
	\begin{thebibliography}{99}  
		
		\bibitem{ref1}Skolem, T. (1958). Some Remarks on the Triple Systems of Steiner. MATHEMATICA SCANDINAVICA, 6, 273–280.
		\bibitem{ref2}D. R. Stinson, Combinatorial Designs: Constructions and Analysis, Springer, 2003
		\bibitem{ref3}吕书龙,梁飞豹,刘文丽.关于评委评分的评价模型[J].福州大学学报(自然科学版),2010,38(03):358-362.
		
	\end{thebibliography}
	\newpage
	\section{附录1}
	
	参考文件列表:
	
	\href{https://github.com/tystys404/2022njumathematicalmodeling}
	
	\section{附录2}
	\paragraph{问题一求解最优分配方案lingo程序}
	~\\
	\begin{figure}[h]
		\centering
		\includegraphics[scale=0.4]{1.jpg}
	\end{figure}
	\paragraph{问题三权重控制代码}
	~\\
		\lstinputlisting[language=matlab]{problem3/quanhzong.m}
	
	~\\
	\paragraph{问题三整体计算流程代码}
	~\\
		\lstinputlisting[language=matlab]{problem3/shumo.m}
	\paragraph{问题三结合权重与数据的计算代码}
	~\\
		\lstinputlisting[language=matlab]{problem3/suafan.m}
	除了支撑材料的文件列表和源程序代码外,附录中还可以包括下面内容:
	\begin{itemize}
		\item 某一问题的详细证明或求解过程;
		\item 自己在网上找到的数据;
		\item 比较大的流程图;
		\item 较繁杂的图表或计算结果
	\end{itemize}
	
\end{document}